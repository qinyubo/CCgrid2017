\label{Section:experiments}
\subsection{Infrastructure overviews}
Caliburn
Caliburn is the new supercomputer from RDI2, which is ranked \#166 worldwide in the Top500 list of June 2016. This system is based on TatTwin SuperServer solution and it has 560 nodes containing 20,160 cores, 140 TB of RAM memory, 218 TB of non-volatile memory and 100 Gbps Omni-Path fabric (OPA) interconnection, which deliver performance of 603 TFLOPS with a peak performance of 677 TFLOPS.  Other details of each node's hardware is given in Table \ref{table:caliburn_hardware_specification}.

\begin{table}[H]
\begin{center}
\begin{tabular}{|l|l|}
	\hline
	\textbf{H/W Type} & \textbf{H/W Detail}\\ \hline
    CPU & 2\* Intel Xeon E5-2695v4\\ 		
    \hline
    CPU frequency & 2.1 GHz\\
    \hline
    \# of Cores & 2\* 18\\
    \hline
    Cache & 2\* 45 MB\\
    \hline
    Memory &\\
    \hline
    Disk & \\
    \hline
    Network bandwidth & \\
    \hline
\end{tabular}
\caption{Caliburn hardware specification}
\label{table:caliburn_hardware_specification}
\end{center}
\end{table}



\subsection{Power measurement}
The Caliburn supports Intelligent Platform Management Interface (IPMI), which is a set of computer interface specifications that is led by Intel\cite{wikiipmi}. This IPMI provides independently management and monitoring capabilities of host system's CPU, firmware and operation system and also allow administrators to manage the system remotely and monitor platform status as well, such as system power supplies, temperatures, fans and ,etc. The Caliburn vendor Supermicro provides a tool called SMCIPMITOOL, which gives power measurement function to users via a command line interface\cite{SMCIPMITOOLuserguide}. In the current system set up, the SMCIPMITOOL can provide us concurrently system-wide total power consumption of two nodes. In this case, all of our experiments are running in even number of nodes. SMCIPMITOOL not give high sample rate, as we test, it give roughly 0.5 Hz sampling rate.  [****need to consult Koji about Caliburn IPMI setup details and SMCIPMITOOL measurement accuracy****]


\subsection{Power capping}
Intel Running Average Power Limit (RAPL) provides a standard interface for measuring and constraining processor and memory power. Based on Intel publication \cite{david2010rapl}, RAPL has combined automatic DVFS and clock throttling techniques, but unlike most popular power capping mechanisms that maintain instantaneous power limits, RAPL estimates and maintains an average power limit over a sliding time window. In some studies\cite{rountree2012beyond,sarood2013optimizing}, RAPL gives high accuracy power capping performance. In our experiment, we use Intel RAPL toolkit, call Intel Power Gadget to set power bound.


\subsection{Performance measurement}
In this study, cannot find a universal standard to quantify the impact for these simulation applications result and performance. Therefore, we propose to quantify its scientific impact from system perspective, specifically, by evaluating CPU utilization, I/O traffic, memory pressure and network traffic. To measure those type of data, we are using two profiling tools: perf and sar. 

Perf is a performance analysis tool in Linux, it offers a rich set of commands to collect and analyze performance and trace data. Sar is a Unix System monitor, it can report on various system loads, to measure memory/paging, device load and network. We use the commands listed in  \ref{table:performance properties} to profiling performance .


\begin{table}[H]
\begin{center}
\begin{tabular}{|l|l|}
	\hline
	\textbf{Property} & \textbf{Command line}\\ \hline
    CPU & perf stat -B\\ 		\hline
    I/O & sar -b\\				\hline
    Memory & sar -r\\			\hline
    Network & sar -n DEV\\      \hline
\end{tabular}
\caption{Performance properties measured and respective command lines }
\label{table:performance properties}
\end{center}
\end{table}



\subsection{Application configurations}
These four simulation applications are simulating different problems, but they all based on AMR algorithm, which uses a hierarchical grid structure, and fill finer patches on the region of interest. We tune their refinement level via their input configuration, to have high, medium and low resolutions. We adjust each application's AMR level of refinement via tuning the parameters in \ref{table:amr_resolution}.


\begin{table}[H]
\begin{center}
\begin{tabular}{|l|l|}
	\hline
	\textbf{Application name} & \textbf{AMR resolution parameter}\\ \hline
    ENZO & MaximumRefinementLevel\\ 		\hline
    FLASH & lrefine\_max\\				\hline
   RAMSES & levelmax\\			\hline
    LMC & amr.max\_level\\      \hline
\end{tabular}
\caption{AMR resolution parameters of each simulation application}
\label{table:amr_resolution}
\end{center}
\end{table}

Some applications have their own I/O libraries dependence. FLASH is using parallel version HDF5, otherwise, the I/O is only called by a single CPU. Network is through Intel Omni-path and the MPI is compiled with OpenMPI-1.10.0-hfi.



\subsection{Methodology}
\subsubsection{Finding appropriate power value, using power cap}
In the previous section, we conclude that the application execution time will decrease as level of refinement/resolution decrease or increase as capping down CPU power. In order to keep application execution time the same with high, medium and low resolution, we first record the execution time of high resolution without implement power cap, and then iteratively capping down CPU power by 10 W for running application in medium or low resolution, until their execution time is matching the high resolution?s execution time within 5\% difference. In this way, we can get the appropriate power cap value for each level of resolution.

\subsubsection{Measuring performance}
This experiment is running on 512 cores across 15 nodes. However, due to the limitation of Perf and Sar, they only can profile a single node. Therefore, we only profile the node that launch the experiment. For example, the experiment is running from node 1 to node 15, then we only measure the performance of node 1. In addition, the power consumption measurement covers all the 15 nodes. SMCIPMITOOL is running on login node, and can measure every node?s power consumption.








