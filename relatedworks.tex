\label{Section:related work}

Energy efficiency has become a critical concern for HPC applications. Many algorithms, energy models and run-time systems have been proposed and developed to obtain energy saving during HPC application execution. DVFS and CPU throttling operations are two practical power saving methods. By leveraging those two methods, Kandalla et al. \cite{kandalla2010designing} designed power-aware algorithms that deliver fine-grained power savings the communication phases of single parallel programming model, typically MPI. At meantime, Li et al. \cite{li2010hybrid} extended this power-aware performance prediction model to hybrid MPI/OpenMP applications which can achieve better energy-efficient execution. In addition, in system level, Rountree et al. \cite{rodero2010investigating} developed a runtime system called Adagio, by doing the critical path analysis, it can determine which tasks may be slowed down and also suitable opportunities to apply DVFS to minimize the performance loss in the parallel execution. Ioannou et al. \cite{ioannou2011phase}  describe a runtime system for the Intel Single-chip Cloud Computer (SCC) processor to detect repeatable communication phases followed by an application of frequency scaling. 

In addition to those power saving work, there also have lots of research works on investigating power management. Lively et al. \cite{lively2011energy}  explored and investigated energy consumption and execution time of MPI only versus hybrid MPI/OpenMP parallel implementations of scientific applications on HPC systems. Gamell et al. \cite{gamell2013exploring} used machine independent code characteristics to build a power model and leverage it to analyze the behavior of simulation workflows and explore data-related energy / performance trade-offs at extreme scales. Rodero et al. \cite{rodero2010investigating} studied the potential of application-centric aggressive power management of HPC scientific workloads and leverage several innovative approaches to tackle the problems in power management mechanisms and controls available at different levels and different subsystems. Power budget could be potentially wasted in processors imbalance usage. A peak power management technique for multi-core systems has been described by Sartori et al cite{sartori2009distributed} , that through choosing the power state for each core to meet the power constraints. Also, several researches study the power budget arrangement. Cebrian et al. \cite{cebrian2011power} proposed that through borrowing power budgets from cores that consume lower power to dynamically adapts the per-core power budgets. While, Gandhi et al. \cite{gandhi2009power} give a power capping strategy to meet the power budget by inserting idle cycles during execution. Moreover, power budget can be constrained by dynamic reconfiguring processors? resources, such as Meng et al. \cite{meng2008multi} proposed to dynamic resizing cache and Konotorinis et al. \cite{kontorinis2009reducing} proposed to reconfigure load-store queues n floating point units and, etc. 

Most of above power management strategies are using DVFS. Since Intel SandyBridge family processors, Intel provides Running Average Power Limit (RAPL) for controlling the power constraint on processors and memory. Several studies have evaluated the RAPL power management system. Zhang et al. \cite{zhang2015quantitative} give a systematic evaluation of RAPL behavior such as stability, setting time, overshoot and, etc. Rountree et al. \cite{rountree2012beyond} through evaluated power consumption for package and memory subsystem to explore RAPL as a replacement for DVFS in HPC systems. In \cite{porterfield2015application}\cite{sarood2013optimizing}, RAPL also been used to study application runtime variability and power optimization for exascale computing.
Measuring the power/energy consumption is the key for energy-aware scheduling. RAPL provides power measurement ability which mechanism and performance have been investigated in \cite{porterfield2015application} \cite{hahnel2012measuring}. Vignesh et al. \cite{adhinarayanan2015greenness} use RAPL to measure the CPUs and RAMs power consumption, in order to study the greenness of the in-situ and the post-processing visualization pipelines and claim that the average RAPL measurement error rate of less than 1\%. 










