\subsubsection{Power measurement methodology}
Our experiment platform CAPER is instrumented with both coarse- and fine-grained power metering at server level. On the one hand, an instrumented Raritan iPDU PX2-4527X2U-K2 provides power measurements at 1 Hz. While, on the other hand, a Yokogawa DL850E ScopeCorder provides voltage and current measurement for all nodes through 1 Ms/s modules, and can sample power data at rate of up to 10 KHz. In addition to those server level measurement, we have RAPL meter to provide power measurement at a up to 20Hz sampling rate in processor level. 


\textbf{Use of RAPL power metering}\\
RAPL meter measuring power through reading Machine-Specific Registers (MSRs). Intel released Power Gadget API for using RAPL. This API is a framework that provides very comprehensive information including reading current estimated processor power, current processor frequency, base frequency, thermal design power (TDP), current temperature, timestamps and etc. Also, Intel gives a ready-to-use software-based power monitoring tool called Intel Power Gadget, this has a completed interface for using RAPL. We are using this to measure the processors power usage. 


\textbf{RAPL meter sampling frequency considerations}\\
Thought, according to Intel’s manual,\cite{intel64and} RAPL MSRs would update at rate of once every 1 ms, using very lower frequency (e.g. less than 20 ms) may result in significant overhead and might also increase the power consumption, which would make the data less meaningful.\cite{usingtheintelpowergadgetapionmacosx} Also, since the instantaneous processor frequency would change very frequently, sampling data may be more useful if you sample often and average the samples overtime. So, Intel recommend sampling frequency of 50 ms or upper. For our experiment, we don’t require high resolution power data, so we choose to sample twice per second.


\textbf{Reading power measurement data from PDU}\\
Raritan PDU keep measuring the whole servers power consumption and writing the power measurement data to power log files with timestamp. CAPER has eight nodes, so we extract real-time power measurement data and the timestamp from each node’s power log file after simulation started using Simple Network Management Protocol (SNMP) queries to the PDU from a side script running in an independent node.


\subsubsection{AMR resolution adjustment}
LMC simulation program is based on AMR algorithm, which uses a hierarchical grid structure, and fill finer patches on the region of interest. Therefore, the simulation’s resolution from the computational point of view is mainly directed from the AMR algorithm resolution configuration. There are many parameters controlling AMR solution, but in the work, only those four parameters listed in Table \ref{table:table_tune_resolution} have been used to tune the simulation resolution. 


\begin{table}[H]
\begin{center}
\begin{tabular}{|l|l|}
	\hline
	\textbf{Parameter} & \textbf{Definition}\\ \hline
    amr.n\_cell & Number of cells in each direction at the coareset level\\ 		\hline
    amr.max\_level & Number of levels of refinement above the coarsest level\\
	\hline
    amr.ref\_ratio & Ratio of coarse to fine grid spacing between subsequent levels\\
    \hline
    amr.regrid\_int & How often to regrid\\
    \hline
\end{tabular}
\caption{AMR parameters which are used to tune simulation resolution}
\label{table:table_tune_resolution}
\end{center}
\end{table}

Each parameter has different impact on the resolution, which is illustrated with the example use case shown in Figure \ref{fig:AMR_resolution_explaination}. The parameters of the example case are discussed as follows.

Example case:
\begin{itemize}
\item amr.n\_cell = 32 32 32
\end{itemize}
This would define the domain size (at coarsest level) to have 32 cells in the x-direction, 32 cells in the y-direction and 32 cells in the z-direction. (If it is in the 2D input file, the last number will be ignored). As shown in Figure \ref{fig:AMR_resolution_explaination}, there are 32 cells in both x and y direction.

\begin{itemize}
\item amr.max\_level = 2
\end{itemize}
This would set a maximum of 2 refinement levels in addition to the coarse level. Within the calculation, the number of refinement level must be $\leqslant$ amr.max\_level, but it can be change in time and it is not necessary always be equal to amr.max\_level. Because these additional levels will only be created if the tagging criteria are such that cells are flagged as needing refinement. Figure \ref{fig:AMR_resolution_explaination} shows the mesh grids with maximum of 2 refinement levels.

\begin{itemize}
\item amr.ref\_ratio = 4 2
\end{itemize}
Refinement ration means how many individual cells will a cell be divided into. For example in the left-hand side Figure \ref{fig:AMR_resolution_explaination}, Setting amr.ref\_ratio = 4 2 means dividing cell into 4 cells from levels 0 and 1, and dividing cell into 2 from levels 1 and 2.

\begin{itemize}
\item amr.regrid\_int = 2
\end{itemize}
This would tell the code to regrid every 2 steps. This means level l+1 grids will be created every 2 level l time steps.


\begin{figure}[H]
	\centering
    \includegraphics[width=8cm]{figs/AMR_resolution_explaination.png}
        \caption{AMR mesh grids outcome of the example configuration}
        \label{fig:AMR_resolution_explaination}
\end{figure}
