The computational demand of high-performance computing (HPC) applications has brought major changes to the HPC system architecture. As a result, it is now possible to run simulations faster and get more accurate results. But behind this, power and energy are becoming critical concerns for HPC systems, e.g. Tianhe-2 has reached speed 33.86 PFLOPS at power of 17.6 MW\cite{wiki_tianhe}, which cost electric around 30 million per year\cite{2014tianhe}. U.S. Department of Energy (DOE) has set the goal to achieve exascale performance with a power budget of 20MW\cite{lucas2014doe}, this make power efficiency become one of the critical challenges for the exascale research.

Current research efforts have studied power and performance tradeoffs, and how to balance these, e.g., using DVFS to meet power constraints, which significantly impacts performance. However, scientific applications may not tolerate degradation in performance and other tradeoffs need to be explored to meet power budgets, e.g., involving the application in making energy-performance tradeoff decisions. 

This research focuses on studying the properties and exploring the performance and power\textbackslash energy tradeoffs of Adaptive Mesh Refinement (AMR) based simulation applications. Our experimental evaluation provides an empirical evaluation of different application configurations that gives insights into the power-performance tradeoffs space for those AMR applications. The key contribution of this work is a better understanding of the running behavior of these AMR-based applications and the power-performance tradeoffs for these applications under power constraints, which can be used to better schedule power budgets across HPC systems.\cite{david2010rapl}

